\documentclass[sigconf,review, anonymous]{acmart}
\acmConference[ISSTA 2022]{ACM SIGSOFT International Symposium on Software Testing and Analysis}{18-22 July, 2022}{Daejeon, South Korea}

\usepackage{authorcomments}

%%
%% end of the preamble, start of the body of the document source.
\begin{document}

\title{{\tt signatr}: Fuzzing for Function Type Signatures}

%%
%% The abstract is a short summary of the work to be presented in the
%% article.
\begin{abstract}
The abstract will go here.
It will be great.
The greatest abstract.

I set up author comments for everyone: \AT{for me}, \PB{for Pierre}, \FK{for Filip}, and \JV{for Jan}, oh and \TODO{for TODOs}.
Feel free to change the colours in {\tt authorcomments.sty}.
\end{abstract}

%%
%% This command processes the author and affiliation and title
%% information and builds the first part of the formatted document.
\maketitle

\section{Introduction}


\section{Background and Motivation}

\subsection{The R Programming Language}

\subsubsection{Types for R}

\subsection{Fuzzing and Test Generation}
\section{Evaluation}

\AT{We should think of research questions we would like to pose and answer.}


\section{Related Work}

\emph{Randoop}~\cite{pacheco2007randoop} is a feedback-driven random test generation tool for Java, though the technique underpinning it is universally applicable; \AT{in fact, we implemented a version of this technique in R as the baseline for our evaluation}.
The technique described in the \emph{Randoop} paper generates sequences of method calls to test classes, and randomly generates arguments for these calls in two ways: for primitives, a random value is selected from a predefined, but user-extensible list, and for reference types a value is selected at random from those which have been seen, and if none are available then {\tt null} is selected.
\AT{While this technique is effective at generating calls in object-oriented languages, it does not work that well in data science languages for reasons that I have yet to come up with.}

%%
%% The next two lines define the bibliography style to be used, and
%% the bibliography file.
\bibliographystyle{ACM-Reference-Format}
\bibliography{fuzzing}


\end{document}
\endinput
%%
%% End of file `sample-sigconf.tex'.
