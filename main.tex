\documentclass[sigconf,review,anonymous]{acmart}

\usepackage{authorcomments}


\newcommand{\tool}{\texttt{signatr}\xspace}
\newcommand{\numFnsCaseStudy}{20\xspace}
\newcommand{\numPkgsScaleStudy}{20\xspace}

%%
%% end of the preamble, start of the body of the document source.
\begin{document}

\title{Fuzzing in Dynamic Languages is Hard}

%%
%% The abstract is a short summary of the work to be presented in the
%% article.
\begin{abstract}
The abstract will go here.
It will be great.
The greatest abstract.

I set up author comments for everyone: \AT{for Alexi}, \PB{for Pierre}, \FK{for Filip}, and \JV{for Jan}, oh and \TODO{for TODOs}.
Feel free to change the colours in {\tt authorcomments.sty}.
\end{abstract}

%%
%% This command processes the author and affiliation and title
%% information and builds the first part of the formatted document.
\maketitle

\section{Introduction}


\section{Background and Motivation}

\subsection{The R Programming Language}

\subsubsection{Types for R}

\subsection{Fuzzing and Test Generation}
\section{Approach}

\subsection{Database of Values}

\begin{itemize}
    \item \AT{We need to discuss the relaxation parameters, since those are central to the fuzzing approach.}
    \item \AT{Also discuss the origin tracking, which is important as input to the fuzzer.}
\end{itemize}

\subsection{Fuzzing Technique}

\begin{itemize}
    \item Say we are fuzzing some function $f$ from a package $p$.
    \item We first consult the database to find all values that were observed as input to $f$; these are our \textit{argument seeds}.
    \item We iteratively generate new calls to $f$ as follows: \ldots first we relax on basically every database parameter \ldots we slowly lower relaxation as the iterations progress (this way, we explore many different parameters at the beginning, and slowly hone-in on values that are likely to work) \ldots
    \item \AT{I'll make this into an algorithm or something.}
\end{itemize}

\subsubsection{Coverage-Guided Fuzzing}

\AT{Idea: parse the code, determine branch conditions that involve parameters; run branch conditions on values in the DB to get different results; pass those in in new calls.}
\section{Implementation}

\AT{Implementation details here.}
\section{Evaluation}

\AT{We should think of research questions we would like to pose and answer.}


%
\section{Case Studies}
\label{sec:case-studies}
\section{Discussion}

\section{Threats to Validity}

\begin{itemize}
    \item \ldots our selection of projects and functions may not be representative \ldots
    \item \AT{others?}
\end{itemize}
\section{Related Work}

\emph{Randoop}~\cite{pacheco2007randoop} is a feedback-driven random test generation tool for Java, though the technique underpinning it is universally applicable; \AT{in fact, we implemented a version of this technique in R as the baseline for our evaluation}.
The technique described in the \emph{Randoop} paper generates sequences of method calls to test classes, and randomly generates arguments for these calls in two ways: for primitives, a random value is selected from a predefined, but user-extensible list, and for reference types a value is selected at random from those which have been seen, and if none are available then {\tt null} is selected.
\AT{While this technique is effective at generating calls in object-oriented languages, it does not work that well in data science languages for reasons that I have yet to come up with.}

%%
%% The next two lines define the bibliography style to be used, and
%% the bibliography file.
\bibliographystyle{ACM-Reference-Format}
\bibliography{fuzzing}


\end{document}
\endinput
%%
%% End of file `sample-sigconf.tex'.
