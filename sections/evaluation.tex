\section{Evaluation}

\AT{Preamble.}
We pose and answer the following research questions:

\begin{enumerate}
    \item How many successful call signatures are discovered with fuzzing vs. by simply observing calls?
    \AT{This establishes that our technique is useful for expanding the type.}
    \item To what degree does expanding the set of successful call signatures improve coverage?
    \AT{This establishes the usefulness of looking specifically for new call signatures w.r.t. improving coverage, the usual metric for fuzzing papers.}
    \item For those functions where fuzzing does not expand code coverage, do the coverage-guided techniques improve coverage and/or discover more signatures?
    \AT{Hopefully, this shows that our efforts to go beyond basic random value selection are fruitful.}
    \item Are the expanded type signatures useful? \AT{Just a thought; not sure how to measure this. Maybe move to discussion.}
    \item How many bugs are found?
\end{enumerate}

\paragraph{Experiment Server} 
We ran all of our experiments on a \AT{prl3 server specs}.
All reported timing information is \AT{averaged over X runs, with standard deviations reported; timed experiments were conducted on a quiet server with few other processes running to minimize interference.}

%
% RQ1
%

\subsection{How many successful call signatures are discovered with fuzzing vs. by simply observing calls?}

\AT{The idea here is to establish that we uncover more signatures with fuzzing.
We have preliminary results suggesting that this is very likely (the stringr test).}

\paragraph{Experimental Design}
\begin{itemize}
    \item We ran \tool on the \TODO{Y} \AT{exported?} functions from \TODO{X} packages.
    \item Recall the general approach of \tool: for a given package function $f$, we first consult the database for the pre-existing calls to $f$, and take the arguments of those calls as seeds for the fuzzer; then, our test generator iteratively generates new inputs by querying the database for values based on the initial seed.
    \item We compare the signatures generated by this process to the signatures corresponding to the pre-existing calls.
\end{itemize}

\paragraph{Results}

\begin{itemize}
    \item \TODO{Table}
\end{itemize}