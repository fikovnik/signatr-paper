\section{Introduction}

\subsection{Notes}

We're going to focus on a study of fuzzing for successful inputs in a dynamic language.
Fuzzers need to understand how to call a function, and that's the space we are exploring.
We have a truly ludicrous amount of signatures, and hopefully we can come up with interesting questions and conclusions.

\subsection{Actual Intro Draft}

\begin{itemize}
    \item The permissive semantics of dynamic languages complicate the task of automatically generating inputs for functions, known as \textit{fuzzing}.
    \item In a dynamic language, functions can successfully execute while exhibiting strange behaviour when supplied with unexpected inputs (inputs that would typically be disallowed in more strict, statically typed languages), so it is difficult for fuzzers to gather much feedback about function execution.
    \item Further complicating matters, static analysis of dynamic code is unlikely to yield meaningful insights, and so there is not much information to leverage ahead-of-time to help guide a fuzzer.
    %
    \item \AT{Something about why we want to fuzz dynamic languages.}
    \item \AT{E.g., in R we want to ensure that Notebooks are robust and correct.}
    \item \AT{E.g., server-side JavaScript is becoming increasingly popular, and fuzzing servers for security vulnerabilities would be of great benefit.}
    %
    % \item \AT{Something about types? Need to lead into the question at the centre of the paper.}
    %
    \item One of the first steps of an automated fuzzing tool is to determine what are valid inputs for a function being fuzzed, and our goal in this paper is to answer the question: \textit{how many successful calls can we find?}
    \item \textit{Grammar-based fuzzing} is a technique in which a fuzzer is equipped with a grammar specifying a language of valid inputs \AT{, but defining these grammars can be cumbersome for programmers and they might not be portable.}
    \item \textit{Mutation-based fuzzing} is another technique, in which a fuzzer will mutate a set of previously ``vetted'' inputs (e.g., inputs that were part of a test suite, inputs that explored a new code path through a function, etc.).
    %
    \item In this work, we explore the space of general-purpose fuzzing in dynamic languages, and identify several key challenges. 
    \item We design a fuzzing approach that attempts to find successful calls to a function, with success defined as a call that did not result in an error and did not generate warnings.
    \item Rather than generating inputs, our technique relies on a database of \AT{millions} of unique values we have seen from executing code, and develop a technique to write efficient and expressive queries over this database.
    \item We implement this approach in a fuzzer called \tool for the R programming language, a highly popular and highly dynamic language.
    %
    \item To reiterate, the goal of this work is to shed some light on how difficult it is to automatically build a baseline level of understanding of dynamic code through fuzzing.
    \item To achieve this, we report on two studies: (1) we run \tool on \numPkgsScaleStudy R packages, and report on the number of successful calls that \tool uncovered, and (2) we conduct a case study on \numFnsCaseStudy R functions, wherein we run \tool for an extended period of time and manually analyze the successful calls as well as the code to build an understanding of what constitutes a ``good'' call.
\end{itemize}